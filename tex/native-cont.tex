
% Frontmatter {{{

\documentclass{report}

\usepackage{verbatim}
\usepackage{amsmath}
\usepackage{amsfonts}
\usepackage{amssymb}
\usepackage{mathrsfs}
\usepackage[a4paper, margin=60pt]{geometry}
\usepackage{parskip}

\frenchspacing

\title{\emph{native-cont}: Reduction Semantics \& Derivations}
\author{L. S. Leary}

% }}}

\begin{document}
\maketitle
\tableofcontents

\chapter{main} % {{{

\section{NativeCont} % {{{

\subsection{Scope} % {{{

Swap incompatibilities!

% Layout
\newcommand{\p}{ \kern0.5pt }
\newcommand{\n}{ \kern-0.5pt }
%
%
% Haskell
\newcommand{\code}[1]{\texttt{#1}}
%
% INCOMPATIBILITY: MathJax doesn't support \expandafter
%
% LaTeX
\newcommand{\declare}[2][]{
  \expandafter\newcommand\csname #1#2\endcsname{\code{#2}}
}
\newcommand{\redeclare}[2][]{
  \expandafter\renewcommand\csname #1#2\endcsname{\code{#2}}
}
%
% MathJax
%\newcommand{\declare}[2][]{ \newcommand{#1#2}{\code{#2}} }
%\newcommand{\redeclare}[2][]{ \renewcommand{#1#2}{\code{#2}} }
%
\newcommand{\Hask}{\text{Hask}}
\newcommand{\unit}{\code{()}}
%
% INCOMPATIBILITY: MathJax doesn't want to escape #.
%
% MathJax
%\newcommand{\hash}{\code{#}}
%
% LaTeX
\newcommand{\hash}{\code{\#}}
%
\renewcommand{\|}{\mathpunct{\,|}}
\newcommand{\app}{\p\p\p\p\p}
\newcommand{\comp}{\circ}
\newcommand{\lam}[2]{\lambda #1 . \, #2}
\newcommand{\del}[2]{\delta  #1 . \, #2}
\newcommand{\bind}{>\n\n\n\n\n\n\n>\n\n\n\n\n\n=}
\newcommand{\mapping}{<\n\n\n\n\n\n\n\&\n\n\n\n\n\n\n>}
\declare{const}
\declare{pure}
\declare{Base}
%
% INCOMPATIBILITY: Spacing around . seems to differ greatly in MathJax.
%
% LaTeX
\newcommand{\qual}{.}
%
% MathJax
%\newcommand{\qual}{\p.\n\n\n}
%
% ST
\declare{newSTRef}
\declare{readSTRef}
\declare{writeSTRef}
\declare{modifySTRef}
%
% Delimited Continuations
\newcommand{\sub}{\leq_{enc}}
\newcommand{\redAx}{ \operatorname*{\mathrel{\longrightarrow}} }
\newcommand{\Not}[1]{\n\rlap{\,\,\,\,/} #1}
\newcommand{\step}{ \operatorname*{\mathrel{\longmapsto}} }
\newcommand{\F}[2]{\sideset{^{#1\n}}{^{#2}}{
  \operatorname{\mathcal F}
}}
\newcommand{\E}{\operatorname*{\mathit E}}
\newcommand{\br}[1]{{\left(#1\right)}}
\newcommand{\sq}[1]{{\left[#1\right]}}
\newcommand{\hole}{\bullet}
\newcommand{\eval}{ \operatorname*{\mathrm{eval}} }
\newcommand{\proj}{\operatorname*{\pi}}
\newcommand{\parto}{\rightharpoonup}
\declare{Cont}
\declare{runCont}
\declare[hs]{lower}
\declare{newLimit}
\declare{impose}
\declare{delimit}
\declare{sunder}
\declare{sever}
\declare{control}
\declare{shift}
\declare{abort}
\declare{return}
%
% Base Monad Coercions
\declare{IO}
\declare{contToIO}
\declare{ioToCont}
\declare{unsafeSTToCont}
\declare{runIO}
\declare{unsafeRunST}

% }}}

% Cont, Limit#, sunder# {{{

\subsection{runCont} % {{{

\( \runCont \) extends \( L_\Hask \) with terms of the form \( \runCont \app e \).
The corresponding reduction axioms are twofold.
\begin{align*}
  \runCont \app (\pure \app h) &\redAx_\Hask h \\
  \runCont \app e &\redAx_\Hask \runCont \app e' \iff e \step e'
\end{align*}

% }}}

\subsection{lower\#} % {{{

\( \hslower\hash \) is a coercion; it's erased before execution and consequently has no role in the reduction semantics.

% }}}

\subsection{contToIO \& ioToCont} % {{{

In implementation, \( \contToIO \) and \( \ioToCont \) are both coercions.
Morally, however, they extend \( L_\IO \) with terms of the form \( \contToIO \app e \).
The corresponding reduction axioms are threefold.
\begin{align*}
  \contToIO \app (\pure \app h)
    &\redAx_\IO \pure \app h \\
  \contToIO \app {\E} \sq{\ioToCont \app i}
    &\redAx_\IO i \bind \lam x { \contToIO \app {{\E} \sq{\pure \app x}} } \\
  \contToIO \app e
    &\redAx_\IO \contToIO \app e' \iff e \step e'
\end{align*}

% }}}

\subsection{unsafeSTToCont} % {{{

Coercion.

% }}}

\subsection{newLimit\#} % {{{

Morally, a \emph{limit binder} under which execution continues.
In a dedicated calculus with terms \( e \) we would have \( \delta l. \, e \) to explicitly bind and lexically scope limits, but in Haskell we can only emulate \( \delta \) with \( \newLimit\hash \lambda \) and wield rank-n types to prevent \( l \) from escaping its scope.
\[
  \newLimit\hash \app \lam l {\pure \app h}
    \redAx \pure \app h
\]

% }}}

\subsection{impose\#} % {{{

Execution continues under the limit.
\[
  \impose\hash \app l \app {\left( \pure \app h \right)}
    \redAx \pure \app h
\]

% }}}

\subsection{sunder\#} % {{{

The sole control operator, defined only when executed under a corresponding limit.
\[
  \impose\hash \app l
    \app {\E_x} \sq{\sunder\hash \app p \app l \app h}
      \redAx h \app \lam y {{\E_x} \sq y}
\]
This pseudo-axiom is \emph{simplified}; it treats \( L_\Hask \) as a macro language over \( L_\Cont \), implicitly reducing an \( h \) term to the corresponding \( e \) term if one exists.
Strictly speaking, however, the RHS must be in \( L_\Cont \) or the reduction axiom is ill formed.
The true axiom is \[
  \impose\hash \app l
    \app {\E_x} \sq{\sunder\hash \app p \app l \app h}
      \redAx {\proj} \br{
        {\eval_\Hask} \br{h \app \lam y {{\E_x} \sq y}}
      }
\] where \begin{align*}
  &\eval_\Hask(\cdot) : L_\Hask \parto L_\Hask^v \\
  &\eval_\Hask(h) = v_h \iff h \step_\Hask^* v_h \\
  &{\proj} \br \cdot : L_\Hask \parto L_\text{Cont} \\
  &{\proj} \br h = e \iff h = e,
\end{align*}
restricted to cases for which both partial functions are defined.

\emph{Reductions for derivative control operators will henceforth be given in simplified, pseudo-axiom form.}

% }}}

% }}}

\subsection{About: This Documentation} % {{{

\subsubsection{Reduction Semantics}

\begin{alignat*}{3}
  &L_\Hask \quad\quad &&h   &&::=
    v_h \| h_1 \app h_2 \| \runCont \app e \| \ldots \\
  &L_\Hask^v          &&v_h &&::=
    \lam x h \| e \| i \| \ldots \\
  &L_\Cont &&e   &&::=
    v_e \| e \bind h \| \del l e \| \impose \app l \app e
        \| \sunder \app l \app h \| \ioToCont \app i \| \ldots \\
  &L_\Cont^v          &&v_e &&::=
    x \| \pure \app h \\
  &L_\IO              &&i   &&::=
    v_i \| i \bind h \| \contToIO \app e \| \ldots \\
  &L_\IO^v            &&v_i &&::=
    x \| \pure \app h
\end{alignat*}

\[
  \E_x ::= \hole
         \| \E_x \bind h
         \| \del l \E_x
         \| \impose \app y \app {\E_x}
\]

\begin{gather*}
  e \redAx e_a \wedge e \redAx e_b \implies e_a = e_b \\
  v_e \Not\redAx e
\end{gather*}

\[
  {\E} \sq e \step {\E} \sq{e'} \iff e \redAx e'
\]

\begin{align*}
  &\eval(\cdot) : L \parto L^v \\
  &\eval(e) = v_e \iff e \step^* v_e
\end{align*}

% }}}

% }}}

\section{Control.Continuation} % {{{

\subsection{newSTRef}

\[
  \newSTRef := \unsafeSTToCont \comp \Base\qual\newSTRef
\]

% }}}

\section{Control.Continuation.Explicit} % {{{

\subsection{runIO}

\[
  \runIO \app p := \hslower \app p \comp \ioToCont
\]

\subsection{delimit}

\[
  \delimit \app k := \delta \lam l {\impose \app l \app \br{k \app l}}
\]

\subsection{sever}

\[
  {\E}_1 \sq{
    \impose \app l \app {\E}_2 \sq{\sever \app p \app l \app h}
  } \step^*
  {\E}_1 \sq{h \app \lam x {\impose \app l \app {\E}_2 \sq x}}
\]

\[
  \sever \app p \app l \app f
    := \sunder \app p \app l \app \lam k {
      f \app \br{\impose \app l \comp k}
    }
\]

\subsection{control}

\[
  {\E}_1 \sq { \impose \app l \app {\E}_2 \sq{\control \app p \app l \app h} }
    \step^*
  {\E}_1 \sq { \impose \app l \app \br{h \app \lam x {{\E}_2 \sq x}} }
\]

\[
  \control \app p \app l \app f
    := \control_0 \app p \app l \app
      {\left( \impose \app l \comp f \right)}
\]

\subsection{shift}

\[
  {\E}_1 \sq { \impose \app l \app {\E}_2 \sq{\shift \app p \app l \app h} }
    \step^*
  {\E}_1 \sq {
    \impose \app l \app \br{h \app \lam x {\impose \app l \app {\E}_2 \sq x}}
  }
\]

\[
  \shift \app p \app l \app f
    := \shift_0 \app p \app l \app \br{\impose \app l \comp f}
\]

\subsection{abort0}

\[
  {\E}_1 \sq{\impose \app l \app {\E}_2 \sq{\abort_0 \app p \app l \app e}}
    \step^*
  {\E}_1 \sq e
\]

\[
  \abort_0 \app p \app l := \control_0 \app p \app l \comp \const
\]

\subsection{abort}

\[
  {\E}_1 \sq{\impose \app l \app {\E}_2 \sq{\abort \app p \app l \app e}}
    \step^*
  {\E}_1 \sq{\impose \app l \app e}
\]

\[
  \abort \app p \app l := \control \app p \app l \comp \const
\]

\subsection{return}

\[
  {\E}_1 \sq{\impose \app l \app {\E}_2 \sq{\return \app p \app l \app h}}
    \step^*
  {\E}_1 \sq{\pure \app h}
\]

\[
  \return \app p \app l := \abort_0 \app p \app l \comp \pure
\]

\subsection{readSTRef}

\[
  \readSTRef \app p := \unsafeRunST \app p \comp \Base\qual\readSTRef
\]

\subsection{writeSTRef}

\[
  \writeSTRef \app p \app r
    := \unsafeRunST \app p \comp \Base\qual\writeSTRef \app r
\]

\subsection{modifySTRef}

\[
  \modifySTRef \app p \app r
    := \unsafeRunST \app p \comp \Base\qual\modifySTRef \app r
\]

\subsection{modifySTRef'}

\[
  \modifySTRef' \app p \app r
    := \unsafeRunST \app p \comp \Base\qual\modifySTRef' \app r
\]

\subsection{unsafeRunST}

\[
  \unsafeRunST \app p := \hslower \app p \comp \unsafeSTToCont
\]

% }}}

\section{Control.Continuation.Inferred} % {{{

\subsection{delimit}

\declare[q]{X}
\declare{reflect}
\declare{evince}
\[
  \delimit \app k := \qX\qual\delimit \app \lam l {
    \reflect \app \br{\evince \app l} \app \br{k \app l}
  }
\]

\subsection{newLimit}

\[
  \newLimit \app k := \qX\qual\newLimit \app \lam l {
    \reflect \app \br{\evince \app l} \app \br{k \app l}
  }
\]

% }}}

% }}}

\chapter{loop} % {{{

\section{Control.Continuation.Loop}

\subsection{loop}

\declare[hs]{loop}
\( \hsloop \app \lambda \) is treated as a \emph{loop binder} under which execution continues.
Contrary to \emph{native-cont} conventions, we reuse \( l \) as a loop metavariable.
In particular, \( l_b \) indicates a loop associated with body \( e_b \). \[
  {\E} \sq{\hsloop \app \lam {l_b} \pure \app \unit}
    \step^*
  {\E} \sq{\hsloop \app \lam {l_b} {e_b}}
\]

\subsection{continue}

\declare{continue}
\[
  {\E}_1 \sq{ \hsloop \app \lam {l_b} {{\E}_2 \sq{\continue \app l_b}} }
    \step^*
  {\E}_1 \sq{\hsloop \app \lam {l_b} {\pure \app \unit}}
\]

\subsection{break}

\declare[hs]{break}
\[
  {\E}_1 \sq{ \hsloop \app \lam l {{\E}_2 \sq{\hsbreak \app l}} }
    \step^*
  {\E}_1 \sq{\pure \app \unit}
\]

% }}}

\chapter{exception} % {{{

\section{Control.Continuation.Exception}

\subsection{catchC}

\declare{catch}
\( \catch_C \app \br{\lam {\p\p\cdot\p\p} \cdots} \app h \) is treated as a \emph{check binder} under which execution continues. \[
  {\E} \sq{\catch_C \app \br{\lam x {\pure \app h_1}} \app h_2}
    \step^*
  {\E} \sq{\pure \app h_1}
\]

\subsection{tryC}

\declare{try}
\declare{Left}
\declare{Right}
\[
  {\E} \sq{\try_C \app \lam x {\pure \app h}}
    \step^*
  {\E} \sq{\pure \app \br{\Right \app h}}
\]

\declare{fmap}
\[
  \try_C \app f
    := \catch_C \app \br{\fmap \app \Right \comp f}
                \app \br{\pure \comp \Left}
\]

\subsection{throwC}

\declare{throw}
\[
  {\E}_1 \sq{\catch_C \app \br{ \lam x {{\E}_2 \sq{\throw_C \app x \app h_1}} } \app h_2}
    \step^*
  {\E}_1 \sq{h_2 \app h_1}
\]

% }}}

\chapter{handler} % {{{

\section{Control.Continuation.Handler.Explicit}

\subsection{yield}

\declare{yield}
\declare{install}
\[
  {\E}_1 \sq{\impose \app l \app {\E}_2 \sq{\yield \app p \app {\left( \install \app l \app h_1 \right)} \app h_2}}
    \step^*
  {\E}_1 \sq{h_1 \app p \app h_2 \app \lam x {\impose \app l \app {\E}_2 \sq x}}
\]

\subsection{yieldP}

\[
  {\E}_1 \sq{\impose \app l \app {\E}_2 \sq{\yield_P \app p \app {\left( \install_P \app l \app h_1 \right)} \app h_2}}
    \step^*
  {\E}_1 \sq{h_1 \app p \app h_2 \app \lam x {\impose \app l \app {\E}_2 \sq x}}
\]

% }}}

\chapter{algebraic} % {{{

\section{Control.Continuation.Algebraic}

\subsection{newEffectC}

\declare{newEffect}
\( \newEffect_C \app op \app \lambda \) is treated as an \emph{effect binder} under which execution continues. \[
  {\E} \sq{\newEffect_C \app op \app \lam x {\pure \app h}}
    \step^*
  {\E} \sq{\pure \app h}
\]

\subsection{effC}

\declare{eff}
\[
  {\E}_1 \sq{ \newEffect_C \app op \app \lam x {{\E}_2 \sq{\eff_C \app x \app \sigma}} }
    \step^*
  {\E}_1 \sq{op \app \br{ \sigma \mapping \lam y {\newEffect_C \app op \app \lam x {{\E}_2 \sq y}} }}
\]

% }}}

\end{document}
